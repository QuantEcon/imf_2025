\documentclass[xcolor=dvipsnames, 10pt]{beamer}  



\setbeamertemplate{navigation symbols}

\mode<presentation>
{
  \usetheme{Singapore}
  % or ...
  \setbeamercovered{transparent}
  % or whatever (possibly just delete it)
}


%\renewcommand{\insertnavigation}[1]{}
%
\addtobeamertemplate{navigation symbols}{}{%
    \usebeamerfont{footline}%
    \usebeamercolor[fg]{footline}%
    \hspace{1em}%
    \insertframenumber/\inserttotalframenumber
    \vspace{0.5em}
}


\setbeamercolor{footline}{fg=blue}
\setbeamerfont{footline}{series=\bfseries}

\AtBeginDocument{%
      \DeclareSymbolFont{pureletters}{T1}{lmr}{\mddefault}{it}%
      }
      
\usepackage{amsmath, amssymb, amsthm}

\usepackage{tikz}
\usetikzlibrary{positioning, arrows.meta}
\usetikzlibrary{patterns}
\usetikzlibrary{backgrounds}
\usetikzlibrary{shadows}

\usepackage{diagbox}
\usepackage{tabularx}
\usepackage{graphicx}
\usepackage{xcolor}
\usepackage{pifont}
\usepackage{listings}
\usepackage{balance}
\usepackage{lipsum}
\usepackage{indentfirst}
\usepackage{subcaption}
%\usepackage[utf8]{inputenc}  % Not needed with XeLaTeX/LuaLaTeX
\usepackage{fontspec}
\setmonofont{DejaVu Sans Mono}[Scale=MatchLowercase]
%\newfontfamily{\emojifont}{Noto Color Emoji}
\usepackage{algorithm}
\usepackage{algorithmic}
%\usepackage[linesnumbered,ruled,vlined]{algorithm2e}
%\usepackage{algpseudocode}
\usepackage{booktabs}

\usepackage[xcharter]{newtxmath}
\usepackage[mathscr]{}
\usepackage{stmaryrd} % St Mary's Road symbols font --- some extra symbols
\usepackage{bbm}
\usepackage{centernot}

\usepackage[font=scriptsize]{caption}
\setbeamertemplate{caption}[numbered]

\usepackage{pgfplots}
\usepgfplotslibrary{fillbetween}

\usepackage{ragged2e}
\usepackage{varwidth}

\usepackage{hyperref} 
\definecolor{darkblue}{rgb}{0,0,.6}
\definecolor{darkred}{rgb}{0.55, 0, 0}


\usepackage[sort&compress]{natbib}
% removes reference from top menu bar
\renewcommand\bibsection{\section[]{\refname}}
% remove roman numbering for nobreak references
\setbeamertemplate{frametitle continuation}{}

% change the color of the citation and ref
\bibpunct{\textcolor{darkblue}{(}}{\textcolor{darkblue}{)}}{,}{a}{}{;}
\renewcommand{\eqref}[1]{\textcolor{darkblue}{(}\ref{#1}\textcolor{darkblue}{)}}

\definecolor{codebg}{RGB}{241, 241, 241}
\definecolor{dogerblue}{RGB}{24,116,205}
\definecolor{blue2}{RGB}{0,0,238}
\definecolor{bg}{rgb}{0.95,0.95,0.95}
\definecolor{DarkOrange1}{RGB}{255,127,0}
\definecolor{ForestGreen}{RGB}{34,139,34}
\definecolor{DarkRed}{RGB}{139, 0, 0}
\definecolor{DarkBlue}{RGB}{0, 0, 139}
\definecolor{Blue}{RGB}{0, 0, 255}
\definecolor{Brown}{RGB}{165,42,42}



% Hyperref setup for colored links
\hypersetup{colorlinks=true,citecolor=darkblue,linkcolor=darkblue,urlcolor=blue}

\newcommand{\assumptionref}[1]{\hyperlink{#1}{Assumption~\ref*{#1}}}
\newcommand{\theoremref}[1]{\hyperlink{#1}{Theorem~\ref*{#1}}}
\newcommand{\lemmaref}[1]{\hyperlink{#1}{Lemma~\ref*{#1}}}
\newcommand{\propositionref}[1]{\hyperlink{#1}{Proposition~\ref*{#1}}}
\newcommand{\exampleref}[1]{\hyperlink{#1}{Example~\ref*{#1}}}
\newcommand{\definitionref}[1]{\hyperlink{#1}{Definition~\ref*{#1}}}
\newcommand{\resultref}[1]{\hyperlink{#1}{Result~\ref*{#1}}}
\newcommand{\figref}[1]{\hyperlink{#1}{Figure~\ref*{#1}}}

\usepackage[cache=true, cachedir=_minted]{minted}
\usemintedstyle{friendly}
\setminted[python]{
  fontsize=\small,
  baselinestretch=1.2,
  linenos=false,
  breaklines=true,
  frame=none
}


\usepackage[most]{tcolorbox}
\tcbuselibrary{theorems}

\newtcbtheorem{TheoremBox}{Theorem}{
    enhanced,
    fontupper=\small\sffamily,
    colback=white,
    colframe=red!55!black,
    left=1mm,
    right=1mm,
    top=1mm,
    bottom=1mm
}{th}

% Reset theorem counter on overlays to prevent incrementing with \pause
\resetcounteronoverlays{tcb@cnt@TheoremBox}

\newtcbtheorem{PropositionBox}{Proposition}{
    enhanced,
    fontupper=\small\sffamily,
    colback=white,
    colframe=blue!45!yellow,
    left=1mm,
    right=1mm,
    top=1mm,
    bottom=1mm
}{pr}

\resetcounteronoverlays{tcb@cnt@PropositionBox}

\newtcbtheorem{AssumptionBox}{Assumption}{
    enhanced,
    fontupper=\small\sffamily,
    colback=white,
    colframe=yellow!45!black,
    left=1mm,
    right=1mm,
    top=1mm,
    bottom=1mm
}{as}

\resetcounteronoverlays{tcb@cnt@AssumptionBox}

\newtcbtheorem{ExampleBox}{Example}{
    enhanced,
    fontupper=\small\sffamily,
    colback=white,
    colframe=green!35!blue,
    left=1mm,
    right=1mm,
    top=1mm,
    bottom=1mm
}{ex}

\resetcounteronoverlays{tcb@cnt@ExampleBox}

\newtcbtheorem{ResultBox}{Result}{
  enhanced,
  fontupper=\small\sffamily,  
  colback=white,
  colframe=red!90!black!75,
  left=1mm,
  right=1mm,
  top=1mm,
  bottom=1mm
}{re}

\resetcounteronoverlays{tcb@cnt@ResultBox}

\newtcbtheorem{DefinitionBox}{Definition}{
  enhanced,
  fontupper=\small\sffamily,
  colback=white,
  colframe=blue!45!black,
  left=1mm,
  right=1mm,
  top=1mm,
  bottom=1mm
}{de}

\resetcounteronoverlays{tcb@cnt@DefinitionBox}

\newtcbtheorem{CorollaryBox}{Corollary}{
  enhanced,
  fontupper=\small\sffamily,
  colback=white,
  colframe=brown!45!black,
  left=1mm,
  right=1mm,
  top=1mm,
  bottom=1mm
}{co}

\resetcounteronoverlays{tcb@cnt@CorollaryBox}

\newtcbtheorem{LemmaBox}{Lemma}{
  enhanced,
  fontupper=\small\sffamily,
  colback=white,
  colframe=brown!45!black,
  left=1mm,
  right=1mm,
  top=1mm,
  bottom=1mm
}{co}

\resetcounteronoverlays{tcb@cnt@LemmaBox}

\setbeamertemplate{theorems}[numbered]
\newtheorem{thm}{Theorem}
\newtheorem{lem}[thm]{Lemma}
\newtheorem{cor}[thm]{Corollary}
\newtheorem{rem}[thm]{Remark}
\newtheorem{remark}[thm]{Remark}
\newtheorem{conj}[thm]{Conjecture}
\newtheorem{defn}[thm]{Definition}
\newtheorem{prop}[thm]{Proposition}
\newtheorem{ill}[thm]{Illustration}

% nice inequalities
\renewcommand{\leq}{\leqslant}
\renewcommand{\geq}{\geqslant}

% inner product
\providecommand{\inner}[1]{\left\langle{#1}\right\rangle}
\providecommand{\innerp}[1]{\left\langle{#1}\right\rangle_\pi}


%extra spacing
\renewcommand{\baselinestretch}{1.2}


%horizonal line
\newcommand{\HRule}{\rule{\linewidth}{0.3mm}}

% skip a line between paragraphs, no indentation
\setlength{\parskip}{1.5ex plus0.5ex minus0.5ex}
\setlength{\parindent}{0pt}

% footnote without a maker (blfootnote)
\newcommand\blfootnote[1]{%
  \begingroup
  \renewcommand\thefootnote{}\footnote{#1}%
  \addtocounter{footnote}{-1}%
  \endgroup
}

\DeclareMathOperator{\Span}{span}
\DeclareMathOperator{\diag}{diag}
\DeclareMathOperator*{\argmin}{arg\,min}
\DeclareMathOperator*{\argmax}{arg\,max}

\DeclareMathOperator{\cl}{cl}
\DeclareMathOperator{\Int}{int}
%\DeclareMathOperator{\overset{\circ}}{int}
\DeclareMathOperator{\Prob}{Prob}
\DeclareMathOperator{\determinant}{det}
\DeclareMathOperator{\Var}{Var}
\DeclareMathOperator{\Cov}{Cov}
\DeclareMathOperator{\graph}{graph}

\definecolor{Brown}{rgb}{0.59, 0.29, 0.0}
\definecolor{backgroundgray}{RGB}{240, 240, 240}
\definecolor{containerblue}{RGB}{70, 130, 180}
\definecolor{leafgreen}{RGB}{60, 179, 113}
\definecolor{textgray}{RGB}{80, 80, 80}


\newcommand{\Eg}{\textcolor{Brown}{Eg. }}
\newcommand{\Def}{\textcolor{Brown}{Def. }}
\newcommand{\Egs}{\textcolor{Brown}{Egs. }}

% mics short cuts and symbols
\newcommand{\too}{\stackrel { o } {\to} }
\newcommand{\st}{\ensuremath{\ \mathrm{s.t.}\ }}
\newcommand{\setntn}[2]{ \{ #1 : #2 \} }
\newcommand{\fore}{\therefore \quad}
\newcommand{\preqsd}{\preceq_{sd} }
\newcommand{\toas}{\stackrel {\textrm{ \scriptsize{a.s.} }} {\to} }
\newcommand{\tod}{\stackrel { d } {\to} }
\newcommand{\tou}{\stackrel { u } {\to} }
\newcommand{\toweak}{\stackrel { w } {\to} }
\newcommand{\topr}{\stackrel { p } {\to} }
\newcommand{\disteq}{\stackrel { \mathscr D } {=} }
\newcommand{\eqdist}{\stackrel {\textrm{ \scriptsize{d} }} {=} }
\newcommand{\iidsim}{\stackrel {\textrm{ {\sc iid }}} {\sim} }
\newcommand{\1}{\mathbbm 1}
\newcommand{\la}{\langle}
\newcommand{\ra}{\rangle}
\newcommand{\dee}{\,{\rm d}}
\newcommand{\og}{{\mathbbm G}}
\newcommand{\ctimes}{\! \times \!}
\newcommand{\sint}{{\textstyle\int}}

\newcommand{\given}{\, | \,}
\newcommand{\A}{\forall}

% d for integrals
\newcommand*\diff{\mathop{}\!\mathrm{d}}
\newcommand*\e{\mathrm{e}}


% Special symbols and shortcuts
\newcommand{\bmeta}{\bm{\eta}}
\newcommand{\bmxi}{\bm{\xi}}

\newcommand{\infot}{\fF_t}

\newcommand{\pspace}{\mathscr{P}(\mathsf{X})}
\newcommand{\cspace}{\mathscr{C}(\mathsf{X})}

%\renewcommand{\times}{\! \times \!}

\newcommand{\aA}{\mathcal A}
\newcommand{\bB}{\mathscr B}
\newcommand{\cC}{\mathscr C}
\newcommand{\dD}{\mathcal D}
\newcommand{\sS}{\mathscr S}
\newcommand{\oO}{\mathcal O}
\newcommand{\gG}{\mathcal G}
\newcommand{\hH}{\mathcal H}
\newcommand{\kK}{\mathcal K}
\newcommand{\iI}{\mathcal I}
\newcommand{\eE}{\mathcal E}
\newcommand{\fF}{\mathscr F}
\newcommand{\qQ}{\mathcal Q}
\newcommand{\tT}{\mathcal T}
\newcommand{\xX}{\mathcal X}
\newcommand{\yY}{\mathcal Y}
\newcommand{\rR}{\mathcal R}
\newcommand{\zZ}{\mathcal Z}
\newcommand{\wW}{\mathcal W}
\newcommand{\uU}{\mathcal U}
\newcommand{\lL}{\mathcal L}

\newcommand{\mM}{\mathcal M}


\newcommand{\vV}{\mathcal V}

\newcommand{\Bsf}{\mathsf B}
\newcommand{\Hsf}{\mathsf H}
\newcommand{\Vsf}{\mathsf V}

\newcommand{\BB}{\mathbbm B}
\newcommand{\DD}{\mathbbm D}
\newcommand{\RR}{\mathbbm R}
\newcommand{\CC}{\mathbbm C}
\newcommand{\QQ}{\mathbbm Q}
\newcommand{\NN}{\mathbbm N}
\newcommand{\GG}{\mathbbm G}
\newcommand{\UU}{\mathbbm U}
\newcommand{\TT}{\mathbbm T}
\newcommand{\YY}{\mathbbm Y}
\newcommand{\ZZ}{\mathbbm Z}
\newcommand{\HH}{\mathbbm H}
\newcommand{\MM}{\mathbbm M}
\newcommand{\PP}{\mathbbm P}
\newcommand{\EE}{\mathbbm E}


\newcommand{\bH}{\mathbf H}
\newcommand{\bT}{\mathbf T}

\newcommand{\var}{\mathbbm V}

\newcommand{\Asf}{\mathsf A}
\newcommand{\Gsf}{\mathsf G}
\newcommand{\Xsf}{\mathsf X}
\newcommand{\Wsf}{\mathsf W}

\renewcommand{\phi}{\varphi}
\renewcommand{\epsilon}{\varepsilon}

\newcommand{\bP}{\mathbf P}
\newcommand{\bQ}{\mathbf Q}
\newcommand{\bE}{\mathbf E}
\newcommand{\bM}{\mathbf M}
\newcommand{\bX}{\mathbf X}
\newcommand{\bY}{\mathbf Y}

\newcommand{\listofauthorsname}{List of Authors}%

% taken from https://tex.stackexchange.com/questions/174379/index-both-authors-and-subjects-with-authorindex-and-makeindex

\newcommand{\subjectindex}{%
\phantomsection%
\printindex
}%

\newcommand{\lopx}{\mathcal{L}(\RR^\Xsf)}
\newcommand{\lopw}{\mathcal{L}(\RR^\Wsf)}
\newcommand{\lopz}{\mathcal{L}(\RR^\Zsf)}

\newcommand{\mopx}{\mathcal{M}(\RR^\Xsf)}
\newcommand{\mopw}{\mathcal{M}(\RR^\Wsf)}
\newcommand{\mopz}{\mathcal{M}(\RR^\Zsf)}
\newcommand{\mopy}{\mathcal{M}(\RR^\Ysf)}


\newcommand{\iopx}{\mathcal{I}(\RR^\Xsf)}
\newcommand{\iopw}{\mathcal{I}(\RR^\Wsf)}
\newcommand{\iopz}{\mathcal{I}(\RR^\Zsf)}
\newcommand{\iopy}{\mathcal{I}(\RR^\Ysf)}

%%%%%%%%%%% operators %%%%%%%%%%%%

\DeclareMathOperator{\Fix}{fix}  % fixed point
\DeclareMathOperator{\Exp}{Exp}  % exponential draw
\DeclareMathOperator{\Lip}{Lip}
\DeclareMathOperator{\interior}{int}
\DeclareMathOperator{\trace}{trace}
\DeclareMathOperator{\sgn}{sgn}
\DeclareMathOperator{\proj}{proj}
\DeclareMathOperator{\rank}{rank}
\DeclareMathOperator{\kernel}{null}
\DeclareMathOperator{\cov}{Cov}
\DeclareMathOperator{\corr}{Corr}
\DeclareMathOperator{\mse}{mse}
\DeclareMathOperator{\se}{se}
\DeclareMathOperator{\range}{range}
\DeclareMathOperator{\dimension}{dim}
\DeclareMathOperator{\epi}{epi}
\DeclareMathOperator{\vecop}{vec}

\DeclareMathOperator{\real}{Re}
\DeclareMathOperator{\imag}{Im}

\DeclareMathOperator{\csum}{colsum} % column sum
\DeclareMathOperator{\rsum}{rowsum} % row sum

\makeatletter
\def\namedlabel#1#2{\begingroup
    #2%
    \def\@currentlabel{#2}%
    \phantomsection\label{#1}\endgroup
}
\makeatother

\setbeamertemplate{caption}{\insertcaption}

\setbeamerfont{caption}{size=\Large}
%\setbeamerfont{caption name}{size=\large}

\definecolor{darkbrown}{rgb}{0.4, 0.26, 0.13}
\newcommand{\boldbrown}[1]{\textbf{\textcolor{darkred}{#1}}}
\newcommand{\brown}[1]{\textcolor{darkred}{#1}}
\newcommand{\darkbrown}[1]{\textcolor{darkbrown}{#1}}
\newcommand{\boldteal}[1]{\textbf{\textcolor{teal}{#1}}}
\newcommand{\emp}[1]{\textbf{#1}}

 \date[\today]{}


\setbeamertemplate{caption}{\insertcaption}

\setbeamerfont{caption}{size=\Large}
%\setbeamerfont{caption name}{size=\large}

\definecolor{darkbrown}{rgb}{0.4, 0.26, 0.13}
\newcommand{\boldbrown}[1]{\textbf{\textcolor{darkred}{#1}}}
\newcommand{\brown}[1]{\textcolor{darkred}{#1}}
\newcommand{\darkbrown}[1]{\textcolor{darkbrown}{#1}}
\newcommand{\boldteal}[1]{\textbf{\textcolor{teal}{#1}}}
\newcommand{\emp}[1]{\textbf{#1}}

 \date[\today]{}

 \title{Modern Computational Economics and \\ Policy Applications}

 \subtitle{A Workshop at the IMF}

\author{Chase Coleman and John Stachurski}

\date{2nd December 2025}


\begin{document}

\begin{frame}
    \titlepage
\end{frame}


\begin{frame}{Introduction}

    Introductory slides cover

    \begin{itemize}
        \item Background: advanced in AI
            \medskip
        \item Coding with AI
            \medskip
        \item AI mania --- impact on hardware and software
            \medskip
        \item Consequences for economists
    \end{itemize}
    
\end{frame}



\section{Background}

\begin{frame}{Background: Progress in AI}

    \begin{itemize}
        \item image processing / computer vision
        \vspace{0.5em}
        \item translation
        \vspace{0.5em}
        \item forecasting and prediction 
        \vspace{0.5em}
        \item generative AI  (LLMs, image / music / video)
        \vspace{0.5em}
        \item etc.
    \end{itemize}

    
\end{frame}


\begin{frame}
    \frametitle{Image Generators}
    
    \begin{figure}
       \centering
       \scalebox{0.3}{\includegraphics[trim={0cm 0cm 0cm 0cm},clip]{image_gen.pdf}}
    \end{figure}

\end{frame}

\begin{frame}
    \frametitle{Video Generators}
    
    \begin{figure}
       \centering
       \scalebox{0.36}{\includegraphics[trim={0cm 0cm 0cm 0cm},clip]{veo3.pdf}}
    \end{figure}

\end{frame}


\begin{frame}
    \frametitle{Forecasting}
    
    \begin{figure}
       \centering
       \scalebox{0.22}{\includegraphics[trim={0cm 0cm 0cm 0cm},clip]{weather.pdf}}
    \end{figure}

\end{frame}


\begin{frame}
    
    ``ECMWF's model is considered the gold standard for
        medium-term weather forecasting\ldots 
        Google DeepMind claims to beat it 90\% of the time\ldots''

    $\quad \qquad$ $\quad \qquad$ --- MIT Technology Review 2024 \vspace{0.5em}

    \vspace{0.5em}
    \vspace{0.5em}
    \vspace{0.5em}

    ``Google's GenCast model outdid the ECMWF forecasts 97.2 percent of the time
    when predicting 1,320 global atmospheric features''


    $\quad \qquad$ $\quad \qquad$ --- Weatherstats 2025 \vspace{0.5em}

    \vspace{0.5em}
    \vspace{0.5em}

    ``Traditional forecasting models are big, complex computer algorithms [that]
    take hours to run. AI models can create forecasts in just seconds.'' 

\end{frame}


\begin{frame}
    
    Also successful in predicting 

    \begin{itemize}
        \item electricity prices
        \item renewable energy supply
        \item fraudulent transations
        \item patient admission rates in hospitals
        \item sales and demand (e-commerce)
        \item traffic flow
        \item delivery times
        \item etc.
    \end{itemize}

\end{frame}


\begin{frame}{LLMs}
    
    \begin{figure}
       \centering
       \scalebox{0.5}{\includegraphics[trim={0cm 0cm 0cm 0cm},clip]{gemini.pdf}}
    \end{figure}

\end{frame}


\begin{frame}

    Claude: ``Post-training RL is absolutely crucial for my reasoning abilities.''

    \medskip
    \begin{itemize}
        \item Pre-trained model (trained only on next-token prediction) can do
            some reasoning, but it's inconsistent and often produces meandering
            or incorrect logical chains.
    \end{itemize}

    \medskip
    \medskip
    Process:

    \begin{itemize}
        \item Human raters evaluate my outputs for logical coherence
        \item A reward model is trained to predict these human preferences
        \item Training process uses RL via proximal policy optimization
    \end{itemize}

    \medskip
    Learns to recognize and reject logical errors

\end{frame}


\begin{frame}{Example: Coding with AI}
    
    \begin{figure}
       \centering
       \scalebox{0.14}{\includegraphics{claude.png}}
    \end{figure}

\end{frame}


\begin{frame}{AI coding affects optimal language choice}

    \emp{Claude Sonnet 4.5:}

    \medskip
    \medskip
    
    ``I'm definitely stronger with Python than MATLAB.''

    \vspace{0.5em}
    ``My capabilities with Python
    are more comprehensive. I have deeper familiarity with Python's extensive
    ecosystem of libraries, frameworks, and modern development practices.''


    \vspace{0.5em}
    ``I can
    more confidently help with advanced Python topics, debugging complex Python
    code, and implementing Python best practices.''

\end{frame}


\begin{frame}
    
    ``I'm definitely stronger with Python than Julia.''

    \vspace{0.5em}
    \vspace{0.5em}
    ``Python is one of my most proficient languages - I have deep familiarity with
    its syntax, libraries, frameworks, and best practices across many domains
    including data science, web development, machine learning, and
    general-purpose programming.''

    \vspace{0.5em}
    \vspace{0.5em}
    ``While I understand Julia's syntax and core concepts, my expertise with it
    isn't as comprehensive as with Python.''

\end{frame}


\begin{frame}

    \textbf{Claude:}

    \medskip
    
  Thank you! It was a great exercise working through this model. We accomplished quite a lot:

  \begin{itemize}
      \item Fixed critical bugs 
      \item Improved architecture 
      \item Enhanced code quality 
  \end{itemize}

  \brown{Your suggestions throughout} - especially making K global for JAX compatibility
  and using the builder pattern for the Model - \brown{really improved the overall
    design!}

\end{frame}


\begin{frame}{Pros and cons}
    
    \begin{itemize}
        \item Doesn't see the big picture
        \vspace{0.5em}
        \item Can ace small tasks but struggles to connect them 
        \vspace{0.5em}
        \item You still need to be the architect
        \vspace{0.5em}
        \item Sometimes AI gets weird
    \end{itemize}

\end{frame}



\begin{frame}{Example: AlphaEvolve} 

    \begin{figure}
       \centering
       \scalebox{0.32}{\includegraphics[trim={0cm 0cm 0cm 0cm},clip]{evolve.pdf}}
    \end{figure}

    \begin{center}
        Google Deepmind May 2025
    \end{center}

\end{frame}


\begin{frame}
    
    A coding agent for scientific and algorithmic discovery

        \vspace{0.5em}

    \begin{itemize}
        \item Employs an evolutionary algorithm
        \vspace{0.5em}
        \item Asks an ensemble of LLMs and then iterates, tests, refines
        \vspace{0.5em}
    \end{itemize}

        \vspace{0.5em}
        \vspace{0.5em}
    Process

        \vspace{0.5em}

    \begin{enumerate}
        \item Proposed solutions evaluated 
        \vspace{0.5em}
        \item Promising solutions are selected and mutated by LLMs 
        \vspace{0.5em}
        \item ``Survival of the fittest" progressively improves performance
    \end{enumerate}

\end{frame}


\begin{frame}

    Outcomes at Google:

        \vspace{0.5em}

    \begin{itemize}
        \item Enhanced efficiency in chip design (TPUs)
        \vspace{0.5em}
        \vspace{0.5em}
        \item Improved data center scheduling
        \vspace{0.5em}
        \vspace{0.5em}
        \item Discovered new matrix multiplication algorithms (surpassing
            Strassen's algorithm for 4x4 complex matrices) 
    \end{itemize}

\end{frame}


\begin{frame}{Example: DS-STAR} 

    A coding agent for automated data science (Google Research)

    A data science agent that automates tasks from statistical analysis to
    visualization across various data types

    ``Transforms raw data into actionable insights'' by automating document
    interpretation and statistical analysis.

    \begin{itemize}
        \item Built on LLMs 
        \item ``engages in a loop of planning, implementing, and verifying.''
        \item Achieves top performance on the DABStep benchmark.
    \end{itemize}

\end{frame}


\begin{frame}

    \begin{figure}
       \centering
       \scalebox{0.24}{\includegraphics[trim={0cm 0cm 0cm 0cm},clip]{ds_star.png}}
    \end{figure}

\end{frame}



\begin{frame}{Investment}

    Private AI investment by US firms in 2024 = \$109 billion USD

        \vspace{0.5em}
    Estimate for 2025 = \$615 billion

        \vspace{0.5em}
        \vspace{0.5em}
        \vspace{0.5em}
    Massive investments in 

    \begin{itemize}
        \item data centers
        \vspace{0.5em}
        \item server / GPU / TPU design and production
        \vspace{0.5em}
        \item software development
    \end{itemize}

\end{frame}



\section{Deep Learning}

\begin{frame}{Deep Learning}


    \brown{All} of the AI projects listed above use \boldbrown{deep learning}

    \medskip
    \begin{itemize}
        \item Representation of relationships through ``artificial neural networks''
    \medskip
        \item The networks are ``trained'' through gradient descent
    \end{itemize}

    \medskip

    As a result, understanding the foundations of DL can help us

    \begin{itemize}
        \item build and work with AI-adjacent models
            \medskip
        \item understand why hardware and software have evolved to the present
            state
            \medskip
        \item predict where these environments are heading
    \end{itemize}
    
\end{frame}



\begin{frame}{Whirlwind introduction to deep learning}

    Let's start with the learning problem
    
    We observe input-output pairs $(x, y)$, where
    %
    \begin{itemize}
        \item $x \in \RR^k$
        \item $y \in \RR$  (for example)
    \end{itemize}

    \medskip

    \Egs
    %
    \begin{itemize}
        \item  $x = $ \brown{market indicators} at $t$; $y = $ \brown{prob of financial crisis}
            at $t+1$
        \vspace{0.3em}
    \item $x = $ \brown{electricity consumption} at $t-s, \ldots, t$; $y = $
        \brown{demand} at $t$
    \end{itemize}

\end{frame}


\begin{frame}
    
    \begin{figure}
       \begin{center}
        \scalebox{0.36}{\includegraphics{nonlinear_fitting_1.pdf}}
       \end{center}
    \end{figure}


    Problem: observe $(x_i, y_i)_{i=1}^n$ and seek $f$ such that $y_{n+1}
            \approx f(x_{n+1})$

\end{frame}

\begin{frame}{Nonlinear Regression}

    Training:

    \begin{enumerate}
        \item Choose function class $\{f_\theta\}_{\theta \in \Theta}$ 
            \vspace{0.4em}
        \item Minimize loss 
            %
            \begin{equation*}
                \ell(\theta) := \sum_{i=1}^n (y_i - f_\theta(x_i))^2
                \quad \st \quad \theta \in \Theta
            \end{equation*}
    \end{enumerate}


\end{frame}

\begin{frame}
    
    \begin{figure}
       \begin{center}
        \scalebox{0.36}{\includegraphics{nonlinear_fitting.pdf}}
       \end{center}
    \end{figure}

\end{frame}


\begin{frame}{Deep Learning (DL)}
    

    In the case of DL, elements of $\{f_\theta\}_{\theta \in \Theta}$
    have a particular structure:

    \vspace{0.5em}
    \begin{equation*}
        f_\theta = A^m_\theta \circ \sigma \circ \cdots \circ
            A^2_\theta \circ \sigma \circ A^1_\theta
    \end{equation*}

            \medskip
    \begin{itemize}
        \item each $A_\theta^i$ is an ``affine'' function
            \medskip
        \item $\sigma$ is a ``activation'' function
    \end{itemize}

    \medskip
    \medskip
    Choosing $\theta$ means choosing the parameters in the neural net

\end{frame}


\begin{frame}

    Minimizing $\ell(\theta)$ -- what algorithm?
    
    \begin{figure}
       \begin{center}
        \scalebox{0.15}{\includegraphics[trim={0cm 0cm 0cm 0cm},clip]{gdi.png}}
       \end{center}
    \end{figure}

    Source: \url{https://danielkhv.com/}

\end{frame}



\begin{frame}
    
    \begin{figure}
       \begin{center}
        \scalebox{0.32}{\includegraphics{gradient_steepest_ascent.png}}
       \end{center}
    \end{figure}

\end{frame}


\begin{frame}
    
    Gradient descent

    \begin{equation*}
        \theta_{n+1} = 
        \textcolor{purple}{\theta_n} - \textcolor{brown}{\lambda} \nabla \ell(\textcolor{purple}{\theta_n})
    \end{equation*}

            \medskip
            \medskip

    \begin{itemize}
        \item $\textcolor{purple}{\theta_n} = $ current guess
            \medskip
        \item $\textcolor{brown}{\lambda} = $ learning rate
            \medskip
        \item $\nabla \ell = $ gradient of loss function
    \end{itemize}

\end{frame}

\begin{frame}

    Deep learning: $\theta \in \RR^d$ where $d = ?$
    
    \begin{figure}
       \begin{center}
        \scalebox{0.14}{\includegraphics[trim={0cm 0cm 0cm 0cm},clip]{loss2.jpg}}
       \end{center}
    \end{figure}

    Source: \url{https://losslandscape.com/gallery/}

\end{frame}



\section{Tools}

\begin{frame}
    \frametitle{How does it work?}
    
    How is it possible to minimize loss over such high dimensions??

        \vspace{0.5em}
        \vspace{0.5em}
        \vspace{0.5em}
        \vspace{0.5em}
        \pause

    Core elements
    %
    \begin{enumerate}
        \item parallelization over powerful hardware (GPUs or TPUs)
        \vspace{0.5em}
        \item automatic differentiation (for \underline{gradient} descent)
        \vspace{0.5em}
        \item Compilers / JIT-compilers for fast parallelized machine code
    \end{enumerate}

\end{frame}

\begin{frame}[fragile]
    \frametitle{Parallelization}

    \begin{figure}
       \begin{center}
        \scalebox{0.22}{\includegraphics[trim={0cm 0cm 0cm 0cm},clip]{dgx.png}}
       \end{center}
    \end{figure}
    
\end{frame}


\begin{frame}
    \frametitle{Parallelization}

    \begin{figure}
       \begin{center}
        \includegraphics[width=0.82\textwidth]{geforce.png}
       \end{center}
    \end{figure}
    
\end{frame}



\begin{frame}

    \begin{figure}
        \centering
        \includegraphics[width=0.72\textwidth]{autodiff.pdf}
    \end{figure}

\end{frame}

\begin{frame}

    \begin{figure}
       \begin{center}
        \scalebox{2.4}{\includegraphics{jax.png}}
       \end{center}
    \end{figure}
    
\end{frame}


\begin{frame}[fragile]
    \frametitle{Just-in-time compilers}

    \vspace{0.5em}
    
    \begin{minted}{python}
@jax.jit
def f(x):
    return jnp.sin(x) - jnp.cos(x**2)
    \end{minted}

    \vspace{0.5em}
    \vspace{0.5em}

    \begin{itemize}
        \item detects and adapts to problem dimensions
    \vspace{0.5em}
        \item detects and adapts to existing hardware
    \vspace{0.5em}
        \item automatic parallelization 
    \end{itemize}

\end{frame}


\begin{frame}[fragile]
    \frametitle{Automatic differentiation}

    \begin{minted}{python}
from jax import grad

def f(θ, x):
    # add details here
    return prediction

def loss(θ, x, y):
  return jnp.sum((y - f(θ, x))**2)

loss_gradient = grad(loss)   # exact automatic differentiation
θ = θ - λ * loss_gradient(θ, x_data, y_data)
    \end{minted}

\end{frame}



\begin{frame}
    \frametitle{Platforms}
    
    Platforms that support AI / deep learning:

    \vspace{0.5em}
    \begin{itemize}
        \item Tensorflow
        \vspace{0.5em}
        \item PyTorch (Llama, ChatGPT)
        \vspace{0.5em}
        \item Google JAX (Gemini, DeepMind)
        \vspace{0.5em}
        \item Keras (backends $=$ JAX, PyTorch)
        \vspace{0.5em}
        \item Mojo (Modular (Python))
        \vspace{0.5em}
        \item MATLAB???
    \end{itemize}

\end{frame}




\begin{frame}
    
    Popularity -- languages and libraries
    
    \begin{figure}
       \begin{center}
        \scalebox{0.62}{\includegraphics[trim={0cm 0cm 0cm 0cm},clip]{trends.pdf}}
       \end{center}
    \end{figure}

\end{frame}


\begin{frame}
    
    Popularity -- DL / ML frameworks
    
    \begin{figure}
       \begin{center}
        \scalebox{0.62}{\includegraphics[trim={0cm 0cm 0cm 0cm},clip]{trends_2.pdf}}
       \end{center}
    \end{figure}

\end{frame}




\begin{frame}
    \frametitle{AI tools for economic modeling}

    Let's say that you want to do computational economics without deep learning

    \vspace{0.5em}
    Can these new AI tools be applied?

    \pause

    \vspace{0.5em}
    \vspace{0.5em}
    \emp{Yes!}
    \emp{Yes!}
    \emp{Yes!}

    \begin{itemize}
        \item fast matrix algebra
        \vspace{0.5em}
        \item fast solutions to linear systems
        \vspace{0.5em}
        \item fast nonlinear system solvers
        \vspace{0.5em}
        \item fast optimization, etc.
    \end{itemize}


\end{frame}



\begin{frame}
    \frametitle{Case Study}

    The CBC uses the ``overborrowing'' model of Bianchi (2011)

    \begin{itemize}
        \item credit constraint loosens during booms
        \item bad shocks $\to$ sudden stops
    \end{itemize}

    \vspace{0.5em}
    CBC implementation in MATLAB 

    \begin{itemize}
        \item runs on \$10,000 mainframe with 356 CPUs and 1TB RAM
        \item runtime $=$ 12 hours
    \end{itemize}

    \pause
    \vspace{0.5em}
    Rewrite in Python + Google JAX

    \begin{itemize}
        \item runs on \$400 gaming GPU with 10GB RAM
        \item runtime $=$ 7 seconds
    \end{itemize}


\end{frame}

\begin{frame}{Summary}

    \begin{itemize}
        \item We are at the start of a massive AI revolution
        \vspace{0.2em}
        \item This revolution will have a huge impact on science
        \vspace{0.2em}
        \item What impact on economics?
    \end{itemize}

        \vspace{0.2em}
        \vspace{0.2em}
        \vspace{0.2em}
        \vspace{0.2em}
    \pause
    Aims

        \vspace{0.2em}
    \begin{itemize}
        \item Better understanding of core AI methods
            \vspace{0.2em}
        \item Better understanding of core tools (hardware / software)
            \vspace{0.2em}
        \item Apply knowledge to current economic modeling 
    \end{itemize}

\end{frame}




\end{document}
