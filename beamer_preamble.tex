

\setbeamertemplate{navigation symbols}

\mode<presentation>
{
  \usetheme{Singapore}
  % or ...
  \setbeamercovered{transparent}
  % or whatever (possibly just delete it)
}


%\renewcommand{\insertnavigation}[1]{}
%
\addtobeamertemplate{navigation symbols}{}{%
    \usebeamerfont{footline}%
    \usebeamercolor[fg]{footline}%
    \hspace{1em}%
    \insertframenumber/\inserttotalframenumber
    \vspace{0.5em}
}


\setbeamercolor{footline}{fg=blue}
\setbeamerfont{footline}{series=\bfseries}

\AtBeginDocument{%
      \DeclareSymbolFont{pureletters}{T1}{lmr}{\mddefault}{it}%
      }
      
\usepackage{amsmath, amssymb, amsthm}

\usepackage{tikz}
\usetikzlibrary{positioning, arrows.meta}
\usetikzlibrary{patterns}
\usetikzlibrary{backgrounds}
\usetikzlibrary{shadows}

\usepackage{diagbox}
\usepackage{tabularx}
\usepackage{graphicx}
\usepackage{xcolor}
\usepackage{pifont}
\usepackage{listings}
\usepackage{balance}
\usepackage{lipsum}
\usepackage{indentfirst}
\usepackage{subcaption}
%\usepackage[utf8]{inputenc}  % Not needed with XeLaTeX/LuaLaTeX
\usepackage{fontspec}
\setmonofont{DejaVu Sans Mono}[Scale=MatchLowercase]
%\newfontfamily{\emojifont}{Noto Color Emoji}
\usepackage{algorithm}
\usepackage{algorithmic}
%\usepackage[linesnumbered,ruled,vlined]{algorithm2e}
%\usepackage{algpseudocode}
\usepackage{booktabs}

\usepackage[xcharter]{newtxmath}
\usepackage[mathscr]{}
\usepackage{stmaryrd} % St Mary's Road symbols font --- some extra symbols
\usepackage{bbm}
\usepackage{centernot}

\usepackage[font=scriptsize]{caption}
\setbeamertemplate{caption}[numbered]

\usepackage{pgfplots}
\usepgfplotslibrary{fillbetween}

\usepackage{ragged2e}
\usepackage{varwidth}

\usepackage{hyperref} 
\definecolor{darkblue}{rgb}{0,0,.6}
\definecolor{darkred}{rgb}{0.55, 0, 0}


\usepackage[sort&compress]{natbib}
% removes reference from top menu bar
\renewcommand\bibsection{\section[]{\refname}}
% remove roman numbering for nobreak references
\setbeamertemplate{frametitle continuation}{}

% change the color of the citation and ref
\bibpunct{\textcolor{darkblue}{(}}{\textcolor{darkblue}{)}}{,}{a}{}{;}
\renewcommand{\eqref}[1]{\textcolor{darkblue}{(}\ref{#1}\textcolor{darkblue}{)}}

\definecolor{codebg}{RGB}{241, 241, 241}
\definecolor{dogerblue}{RGB}{24,116,205}
\definecolor{blue2}{RGB}{0,0,238}
\definecolor{bg}{rgb}{0.95,0.95,0.95}
\definecolor{DarkOrange1}{RGB}{255,127,0}
\definecolor{ForestGreen}{RGB}{34,139,34}
\definecolor{DarkRed}{RGB}{139, 0, 0}
\definecolor{DarkBlue}{RGB}{0, 0, 139}
\definecolor{Blue}{RGB}{0, 0, 255}
\definecolor{Brown}{RGB}{165,42,42}



% Hyperref setup for colored links
\hypersetup{colorlinks=true,citecolor=darkblue,linkcolor=darkblue,urlcolor=blue}

\newcommand{\assumptionref}[1]{\hyperlink{#1}{Assumption~\ref*{#1}}}
\newcommand{\theoremref}[1]{\hyperlink{#1}{Theorem~\ref*{#1}}}
\newcommand{\lemmaref}[1]{\hyperlink{#1}{Lemma~\ref*{#1}}}
\newcommand{\propositionref}[1]{\hyperlink{#1}{Proposition~\ref*{#1}}}
\newcommand{\exampleref}[1]{\hyperlink{#1}{Example~\ref*{#1}}}
\newcommand{\definitionref}[1]{\hyperlink{#1}{Definition~\ref*{#1}}}
\newcommand{\resultref}[1]{\hyperlink{#1}{Result~\ref*{#1}}}
\newcommand{\figref}[1]{\hyperlink{#1}{Figure~\ref*{#1}}}

\usepackage[cache=true, cachedir=_minted]{minted}
\usemintedstyle{friendly}
\setminted[python]{
  fontsize=\small,
  baselinestretch=1.2,
  linenos=false,
  breaklines=true,
  frame=none
}


\usepackage[most]{tcolorbox}
\tcbuselibrary{theorems}

\newtcbtheorem{TheoremBox}{Theorem}{
    enhanced,
    fontupper=\small\sffamily,
    colback=white,
    colframe=red!55!black,
    left=1mm,
    right=1mm,
    top=1mm,
    bottom=1mm
}{th}

% Reset theorem counter on overlays to prevent incrementing with \pause
\resetcounteronoverlays{tcb@cnt@TheoremBox}

\newtcbtheorem{PropositionBox}{Proposition}{
    enhanced,
    fontupper=\small\sffamily,
    colback=white,
    colframe=blue!45!yellow,
    left=1mm,
    right=1mm,
    top=1mm,
    bottom=1mm
}{pr}

\resetcounteronoverlays{tcb@cnt@PropositionBox}

\newtcbtheorem{AssumptionBox}{Assumption}{
    enhanced,
    fontupper=\small\sffamily,
    colback=white,
    colframe=yellow!45!black,
    left=1mm,
    right=1mm,
    top=1mm,
    bottom=1mm
}{as}

\resetcounteronoverlays{tcb@cnt@AssumptionBox}

\newtcbtheorem{ExampleBox}{Example}{
    enhanced,
    fontupper=\small\sffamily,
    colback=white,
    colframe=green!35!blue,
    left=1mm,
    right=1mm,
    top=1mm,
    bottom=1mm
}{ex}

\resetcounteronoverlays{tcb@cnt@ExampleBox}

\newtcbtheorem{ResultBox}{Result}{
  enhanced,
  fontupper=\small\sffamily,  
  colback=white,
  colframe=red!90!black!75,
  left=1mm,
  right=1mm,
  top=1mm,
  bottom=1mm
}{re}

\resetcounteronoverlays{tcb@cnt@ResultBox}

\newtcbtheorem{DefinitionBox}{Definition}{
  enhanced,
  fontupper=\small\sffamily,
  colback=white,
  colframe=blue!45!black,
  left=1mm,
  right=1mm,
  top=1mm,
  bottom=1mm
}{de}

\resetcounteronoverlays{tcb@cnt@DefinitionBox}

\newtcbtheorem{CorollaryBox}{Corollary}{
  enhanced,
  fontupper=\small\sffamily,
  colback=white,
  colframe=brown!45!black,
  left=1mm,
  right=1mm,
  top=1mm,
  bottom=1mm
}{co}

\resetcounteronoverlays{tcb@cnt@CorollaryBox}

\newtcbtheorem{LemmaBox}{Lemma}{
  enhanced,
  fontupper=\small\sffamily,
  colback=white,
  colframe=brown!45!black,
  left=1mm,
  right=1mm,
  top=1mm,
  bottom=1mm
}{co}

\resetcounteronoverlays{tcb@cnt@LemmaBox}

\setbeamertemplate{theorems}[numbered]
\newtheorem{thm}{Theorem}
\newtheorem{lem}[thm]{Lemma}
\newtheorem{cor}[thm]{Corollary}
\newtheorem{rem}[thm]{Remark}
\newtheorem{remark}[thm]{Remark}
\newtheorem{conj}[thm]{Conjecture}
\newtheorem{defn}[thm]{Definition}
\newtheorem{prop}[thm]{Proposition}
\newtheorem{ill}[thm]{Illustration}

% nice inequalities
\renewcommand{\leq}{\leqslant}
\renewcommand{\geq}{\geqslant}

% inner product
\providecommand{\inner}[1]{\left\langle{#1}\right\rangle}
\providecommand{\innerp}[1]{\left\langle{#1}\right\rangle_\pi}


%extra spacing
\renewcommand{\baselinestretch}{1.2}


%horizonal line
\newcommand{\HRule}{\rule{\linewidth}{0.3mm}}

% skip a line between paragraphs, no indentation
\setlength{\parskip}{1.5ex plus0.5ex minus0.5ex}
\setlength{\parindent}{0pt}

% footnote without a maker (blfootnote)
\newcommand\blfootnote[1]{%
  \begingroup
  \renewcommand\thefootnote{}\footnote{#1}%
  \addtocounter{footnote}{-1}%
  \endgroup
}

\DeclareMathOperator{\Span}{span}
\DeclareMathOperator{\diag}{diag}
\DeclareMathOperator*{\argmin}{arg\,min}
\DeclareMathOperator*{\argmax}{arg\,max}

\DeclareMathOperator{\cl}{cl}
\DeclareMathOperator{\Int}{int}
%\DeclareMathOperator{\overset{\circ}}{int}
\DeclareMathOperator{\Prob}{Prob}
\DeclareMathOperator{\determinant}{det}
\DeclareMathOperator{\Var}{Var}
\DeclareMathOperator{\Cov}{Cov}
\DeclareMathOperator{\graph}{graph}

\definecolor{Brown}{rgb}{0.59, 0.29, 0.0}
\definecolor{backgroundgray}{RGB}{240, 240, 240}
\definecolor{containerblue}{RGB}{70, 130, 180}
\definecolor{leafgreen}{RGB}{60, 179, 113}
\definecolor{textgray}{RGB}{80, 80, 80}


\newcommand{\Eg}{\textcolor{Brown}{Eg. }}
\newcommand{\Def}{\textcolor{Brown}{Def. }}
\newcommand{\Egs}{\textcolor{Brown}{Egs. }}

% mics short cuts and symbols
\newcommand{\too}{\stackrel { o } {\to} }
\newcommand{\st}{\ensuremath{\ \mathrm{s.t.}\ }}
\newcommand{\setntn}[2]{ \{ #1 : #2 \} }
\newcommand{\fore}{\therefore \quad}
\newcommand{\preqsd}{\preceq_{sd} }
\newcommand{\toas}{\stackrel {\textrm{ \scriptsize{a.s.} }} {\to} }
\newcommand{\tod}{\stackrel { d } {\to} }
\newcommand{\tou}{\stackrel { u } {\to} }
\newcommand{\toweak}{\stackrel { w } {\to} }
\newcommand{\topr}{\stackrel { p } {\to} }
\newcommand{\disteq}{\stackrel { \mathscr D } {=} }
\newcommand{\eqdist}{\stackrel {\textrm{ \scriptsize{d} }} {=} }
\newcommand{\iidsim}{\stackrel {\textrm{ {\sc iid }}} {\sim} }
\newcommand{\1}{\mathbbm 1}
\newcommand{\la}{\langle}
\newcommand{\ra}{\rangle}
\newcommand{\dee}{\,{\rm d}}
\newcommand{\og}{{\mathbbm G}}
\newcommand{\ctimes}{\! \times \!}
\newcommand{\sint}{{\textstyle\int}}

\newcommand{\given}{\, | \,}
\newcommand{\A}{\forall}

% d for integrals
\newcommand*\diff{\mathop{}\!\mathrm{d}}
\newcommand*\e{\mathrm{e}}


% Special symbols and shortcuts
\newcommand{\bmeta}{\bm{\eta}}
\newcommand{\bmxi}{\bm{\xi}}

\newcommand{\infot}{\fF_t}

\newcommand{\pspace}{\mathscr{P}(\mathsf{X})}
\newcommand{\cspace}{\mathscr{C}(\mathsf{X})}

%\renewcommand{\times}{\! \times \!}

\newcommand{\aA}{\mathcal A}
\newcommand{\bB}{\mathscr B}
\newcommand{\cC}{\mathscr C}
\newcommand{\dD}{\mathcal D}
\newcommand{\sS}{\mathscr S}
\newcommand{\oO}{\mathcal O}
\newcommand{\gG}{\mathcal G}
\newcommand{\hH}{\mathcal H}
\newcommand{\kK}{\mathcal K}
\newcommand{\iI}{\mathcal I}
\newcommand{\eE}{\mathcal E}
\newcommand{\fF}{\mathscr F}
\newcommand{\qQ}{\mathcal Q}
\newcommand{\tT}{\mathcal T}
\newcommand{\xX}{\mathcal X}
\newcommand{\yY}{\mathcal Y}
\newcommand{\rR}{\mathcal R}
\newcommand{\zZ}{\mathcal Z}
\newcommand{\wW}{\mathcal W}
\newcommand{\uU}{\mathcal U}
\newcommand{\lL}{\mathcal L}

\newcommand{\mM}{\mathcal M}


\newcommand{\vV}{\mathcal V}

\newcommand{\Bsf}{\mathsf B}
\newcommand{\Hsf}{\mathsf H}
\newcommand{\Vsf}{\mathsf V}

\newcommand{\BB}{\mathbbm B}
\newcommand{\DD}{\mathbbm D}
\newcommand{\RR}{\mathbbm R}
\newcommand{\CC}{\mathbbm C}
\newcommand{\QQ}{\mathbbm Q}
\newcommand{\NN}{\mathbbm N}
\newcommand{\GG}{\mathbbm G}
\newcommand{\UU}{\mathbbm U}
\newcommand{\TT}{\mathbbm T}
\newcommand{\YY}{\mathbbm Y}
\newcommand{\ZZ}{\mathbbm Z}
\newcommand{\HH}{\mathbbm H}
\newcommand{\MM}{\mathbbm M}
\newcommand{\PP}{\mathbbm P}
\newcommand{\EE}{\mathbbm E}


\newcommand{\bH}{\mathbf H}
\newcommand{\bT}{\mathbf T}

\newcommand{\var}{\mathbbm V}

\newcommand{\Asf}{\mathsf A}
\newcommand{\Gsf}{\mathsf G}
\newcommand{\Xsf}{\mathsf X}
\newcommand{\Wsf}{\mathsf W}

\renewcommand{\phi}{\varphi}
\renewcommand{\epsilon}{\varepsilon}

\newcommand{\bP}{\mathbf P}
\newcommand{\bQ}{\mathbf Q}
\newcommand{\bE}{\mathbf E}
\newcommand{\bM}{\mathbf M}
\newcommand{\bX}{\mathbf X}
\newcommand{\bY}{\mathbf Y}

\newcommand{\listofauthorsname}{List of Authors}%

% taken from https://tex.stackexchange.com/questions/174379/index-both-authors-and-subjects-with-authorindex-and-makeindex

\newcommand{\subjectindex}{%
\phantomsection%
\printindex
}%

\newcommand{\lopx}{\mathcal{L}(\RR^\Xsf)}
\newcommand{\lopw}{\mathcal{L}(\RR^\Wsf)}
\newcommand{\lopz}{\mathcal{L}(\RR^\Zsf)}

\newcommand{\mopx}{\mathcal{M}(\RR^\Xsf)}
\newcommand{\mopw}{\mathcal{M}(\RR^\Wsf)}
\newcommand{\mopz}{\mathcal{M}(\RR^\Zsf)}
\newcommand{\mopy}{\mathcal{M}(\RR^\Ysf)}


\newcommand{\iopx}{\mathcal{I}(\RR^\Xsf)}
\newcommand{\iopw}{\mathcal{I}(\RR^\Wsf)}
\newcommand{\iopz}{\mathcal{I}(\RR^\Zsf)}
\newcommand{\iopy}{\mathcal{I}(\RR^\Ysf)}

%%%%%%%%%%% operators %%%%%%%%%%%%

\DeclareMathOperator{\Fix}{fix}  % fixed point
\DeclareMathOperator{\Exp}{Exp}  % exponential draw
\DeclareMathOperator{\Lip}{Lip}
\DeclareMathOperator{\interior}{int}
\DeclareMathOperator{\trace}{trace}
\DeclareMathOperator{\sgn}{sgn}
\DeclareMathOperator{\proj}{proj}
\DeclareMathOperator{\rank}{rank}
\DeclareMathOperator{\kernel}{null}
\DeclareMathOperator{\cov}{Cov}
\DeclareMathOperator{\corr}{Corr}
\DeclareMathOperator{\mse}{mse}
\DeclareMathOperator{\se}{se}
\DeclareMathOperator{\range}{range}
\DeclareMathOperator{\dimension}{dim}
\DeclareMathOperator{\epi}{epi}
\DeclareMathOperator{\vecop}{vec}

\DeclareMathOperator{\real}{Re}
\DeclareMathOperator{\imag}{Im}

\DeclareMathOperator{\csum}{colsum} % column sum
\DeclareMathOperator{\rsum}{rowsum} % row sum

\makeatletter
\def\namedlabel#1#2{\begingroup
    #2%
    \def\@currentlabel{#2}%
    \phantomsection\label{#1}\endgroup
}
\makeatother

\setbeamertemplate{caption}{\insertcaption}

\setbeamerfont{caption}{size=\Large}
%\setbeamerfont{caption name}{size=\large}

\definecolor{darkbrown}{rgb}{0.4, 0.26, 0.13}
\newcommand{\boldbrown}[1]{\textbf{\textcolor{darkred}{#1}}}
\newcommand{\brown}[1]{\textcolor{darkred}{#1}}
\newcommand{\darkbrown}[1]{\textcolor{darkbrown}{#1}}
\newcommand{\boldteal}[1]{\textbf{\textcolor{teal}{#1}}}
\newcommand{\emp}[1]{\textbf{#1}}

 \date[\today]{}
